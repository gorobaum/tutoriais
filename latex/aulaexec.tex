\documentclass[12pt,onecolumn]{article}
\usepackage[brazilian]{babel}
\usepackage[utf8]{inputenc}
\usepackage{amsmath}
\begin{document}

\title{Aula Exercicio de Introducao ao \LaTeX}
\author{Thiago de Gouveia Nunes}
\maketitle

\section{Introducao ao \LaTeX}
    LaTeX é um programa de formatação de texto e uma expansão do programa
    TeX, criado por Donald Knuth. Mas o que é um programa de
    formatação de texto?

    A maioria dos processadores de texto cuidam de 4 estágios para preparar
    um texto:
    1. O texto entra no computador.
    2. O texto é formatado em linhas, paragráfos e páginas.
    3. O texto é impresso na tela.
    4. O texto é impresso.

    O LaTeX e o TeX se preocupao somente com o segundo estagio
    do processamento. Para formatar um texto usando o LaTeX, escrevemos
    o texto em um editor de texto e compilamos. A partir daí, o \LaTeX
    cuidára da formatação do texto.

\subsection{Tipos de Documentos}
    Os tipos mais comuns de documentos que o \LaTeX aceita são
    \begin{enumerate}
    \item Book
    \item Article
    \item Letter
    \item Report
    \end{enumerate}

\subsection{Fontes}

    Podemos mudar o tamanho e o estilo das fontes usando os seguintes
    comandos:
    \begin{itemize}
    \item \verb#\textrm{...}# \textrm{roman}
    \item \verb#\textsf{...}# \textsf{sans serif}
    \item \verb#\texttt{...}# \texttt{typewriter}
    \item \verb#\textmd{...}# \textmd{medium}
    \item \verb#\textbf{...}# \textbf{boldface}
    \item \verb#\textup{...}# \textup{upright}
    \item \verb#\textit{...}# \textit{italic}
    \item \verb#\textsl{...}# \textsl{slanted}
    \item \verb#\textsc{...}# \textsc{small cap}
    \end{itemize}

\section{Primeiro arquivo}
    Agora, vamos criar um arquivo simples no \LaTeX. Copie as
    linhas abaixo no seu editor de texto favorito e rode o
    compilador de \LaTeX para gerar o pdf.

    \begin{verbatim}
    \documentclass{article}
    \begin{document}
    Hello, world!
    \end{document}
    \end{verbatim}

    Os comandos nesse arquivo são:
    \begin{itemize}
    \item \verb#\documentclass{..}# é o comando
        que define qual o tipo de documento estamos escrevendo.
    \item \verb#\begin{...} e \end{...}# são os
        comandos que definem quando algo começa e termina. No nosso caso,
        eles definem quando o documento começa e termina.
    \end{itemize}

    Agora, vamos acrescentar novos elementos nesse documento,
    e transformá-lo em um documento real.

\section{Estrutura de um Documento}
\subsection{Modificações para o Documento}
    Vamos estudar mais detalhadamente a estrutura de um documento no \LaTeX.
    Já vimos acima o \verb#\documentclass[...]{...}#. Ele aceita várias
    opções para modificar o documento. Essas opções, que colocadas
    entre os colchetes, são:
    \begin{itemize}
    \item \verb#\documentclass[10pt]{article}# Assim, o tamanho das letras do
        documento é setado para 10pt.
    \item \verb#documentclass[letterpaper]{article}# Esse comando faz com que
        o \LaTeX molde o seu texto para ser impresso em papel de carta.
    \item \verb#documentclass[twocolumn]{article}# Esse aqui faz com que o
        texto seja dividido em duas colunas ( como nos dicionários ).
    \item \verb#documentclass[oneside]{article}# Quando o \LaTeX montar o
        texto para impressão, ele vai configurá-lo para imprimir somente
        nas páginas ímpares.
    \end{itemize}

\subsection{Modificações para Páginas}
    Podemos modificar todas as páginas de um documento, ou alguma página
    especifica, usando os comandos \verb#\pagestyle{...}# e o
    \verb#\thispagestyle{...}#, respectivamente.

    Os argumentos que eles aceitam são:
    \begin{itemize}
    \item plain - O cabeçálho fica vazio e o rodapé contêm o número da
    página. Esse é o padrao para o article,
    \item empty - Tanto o cabeçálho como o rodapé ficam vazios,
    \item headings - O rodapé fica vazio e o cabeçálho contem o número da
            página e o nome do capítulo ou seção ou subseção,
    \item myheadings - Igual ao headings, mas o que é mostrando no lugar do
            nome do capítulo pode ser configurado.
    \end{itemize}

    Para modificar o modo com que as páginas são númeradas, usamos o
    comando \verb#\pagenumbering{...}#. Seus argumentos são:
    \begin{itemize}
    \item arabic - Numerais arábicos
    \item roman - Numerais romanos em minúsculo,
    \item Roman - Numerais romanos em maiúsculo,
    \item alph - Numerais por extenso em inglês em minúsculo,
    \item Alph - Numerais por extenso em inglês em maiúsculo.
    \end{itemize}

\subsection{Atualizar Exemplo}
    Esse são só alguns exemplos do que podemos mudar nos nossos documentos.
    Agora, vamos ver como fica o nosso documento de teste se colocarmos
    as opções \emph{twocolumn} e \emph{a4paper} para o documento, opções
    \emph{headings} e \emph{Roman} para as páginas. E vamos
    incrementá-lo com dois paragráfos de texto.

    Para criar novos paragráfos no \LaTeX só precisamos deixar uma linha
    totalmente em branco entre os dois paragráfos.

\subsection{Título}
    O \LaTeX também cuida da criação do título do documento. Para isso
    usamos os seguintes comandos
    \begin{verbatim}
    \title{título}
    \author{autor}
    \date{data}
    \maketitle
    \end{verbatim}

    Note que é preciso usar o comando \verb#\maketitle# ou o título não
    será gerado.

\subsection{Resumo}
    Nos documentos do tipo article e report podemos gerar um resumo usando
    os comandos \verb#\begin{abstract}# e \verb#\end{abstract}#.
    Tudo o que estiver escrito dentro dessas tags vai virar o resumo do
    documento, e será posicionado entre o título e o resto do documento.

\subsection{Divisão do Documento}
    Quando escrevemos textos grandes ( como esse ), é bom criar seções e
    subções para deixar tudo organizado e para facilitar a busca de infor
    mações. O \LaTeX tem uma hierarquia de comandos para criar divisões:
    \begin{enumerate}
	\item \verb#\chapter#
    \item \verb#\section#
    \item \verb#\subsection#
    \item \verb#\subsubsection#
    \item \verb#\paragraph#
    \item \verb#\subparagraph#
    \end{enumerate}
    
    O \verb#\chapter# só pode ser usado no tipo book.

\subsection{Nova Atualização}
    Agora vamos colocar mais informação no nosso documento. Vamos criar duas
    seções com duas subseções cada. Vamos criar um resumo também.

\section{Fórmulas Matemáticas}
    O \TeX foi criado pelo Donal Kunth principalmente para processar
    fórmulas matemáticas. Para colocar expressões matemáticas em linha
    usamos o símbolo \$. Para produzir
        
        A raiz de $ax+b=0$ é $-b/a$.
        
    Devemos escrever no arquivo-fonte
        
        \verb#A raíz de $ax+b=0$ é $-b/a$.#
    
    Esse método é antigo, e é usado pelo \TeX. O \LaTeX
    usa dois padrões para reconhecer expressões

    \begin{verbatim}
    \(...\) 
    ou
    \begin{math} ... \end{math}
    \end{verbatim}
    
    Então, para gerar a mesma linha acima, podemos usarmos
    
    \verb#A raiz de \(ax+b=0\) é \(-b/a\).#
    
\subsection{Expoentes e Índices}
    Vamos usar exemplos para entender como criar expoentes e índices.
    Para produzir expoentes, usamos o símbolo \verb#^#.
    
    Para produzir esse trecho
    $$
    x^n + y^n = z^n
    $$
    devemos escrever \verb#x^n + y^n = z^n#.
    
    O trecho, 
    $$
    (x^m)^n = x^{mn}
    $$
    é gerado pelo código \verb#(x^m)^n = x^{mn}#.
    
    Podemos fazer potência de potências, por exemplo,
    $$
    2^{2^n}+1
    $$
    é produzido pela linha \verb#2^{2^n}+1#.
    
    Agora vamos falar de índices. Para gerar um índice, usamos
    o símbolo \verb#_#.
    
    O trecho abaixo
    
    A sequência $(x_n)$ definida por
    $$
    x_1=1,\quad x_2=1,\quad x_n=x_{n-1}+x_{n-2}\;\;(n>2)
    $$
    é chamada de Fibonacci.
    
    é feita digitando a linha   
\begin{verbatim}
A sequência $(x_n)$ definida por
$$ 
x_1=1,\quad x_2=1,\quad x_n=x_{n-1}+x_{n-2}\;\;(n>2)
$$
é chamada de Fibonacci.
\end{verbatim}.

    Podemos juntar potências e índices usando a seguinte linha de código
    \verb#$(x_nˆ2)$#, com resultado, $(x_n^2)$.
    
\subsection{Raizes}
    Raizes são geradas usando o comando \verb#$\sqrt$#. \verb#$\sqrt{2}$# 
    gera $\sqrt{2}$. Para fazer raizes com raizes não quadradas usamos,
    por exemplo, \verb#$\sqrt[4]{2}$#, que gera $\sqrt[4]{2}$. Podemos
    colocar qualquer fórmula matemática dentro das chaves que o \LaTeX
    vai arrumar a raiz para englobar a fórmula.

\section{Mais Matemática}
    Essa seção vai explorar o pacote \emph{amsmath}. Para incluilo no
    \LaTeX, é só colocar a linha \verb#\usepackage{amsmath}# abaixo da
    linha \verb#\documentclass{...}# no início do arquivo fonte.
   % \subsection{Equações}
   % Como visto acima, podemos usar o símbolo \$ para gerar texto
   % matemático, mas o \emph{amsmath} tem o comando \verb#\begin{equation*}#
   % e \verb#\end{equation)}#, que fazem com que tudo que seja escrito entre
   % eles vire linguagem matemática.
   % Vamos ver um exemplo
    
\subsection{Matriz}
    Podemos fazer um sistema de equações usando o comando \verb#\begin{align*}#.
    Um exemplo:
        
        \begin{align*}
            x+y-z=1\\
            x-y+z=1\\
            x+y+z=1
        \end{align*}
        
    É gerado usando
    
    \begin{verbatim}
    \begin{align*}
        x+y-z=1\\
        x-y+z=1\\
        x+y+z=1
    \end{align*}
    \end{verbatim}

    Para escrever esse sistema como matriz, usamos esses comandos:
    
    \begin{verbatim}
        \begin{equation*}
            \begin{pmatrix}
                1 & 1 & -1\\
                1 & -1 & 1\\
                1 & 1 & 1
             \end{pmatrix}
             \begin{pmatrix}
                x\\
                y\\
                z
             \end{pmatrix}
             =
             \begin{pmatrix}
                1\\
                1\\
                1
             \end{pmatrix}.
        \end{equation*}
    \end{verbatim}
    
    Com esse resultado:
    
    \begin{equation*}
        \begin{pmatrix}
            1 & 1 & -1\\
            1 & -1 & 1\\
            1 & 1 & 1
         \end{pmatrix}
         \begin{pmatrix}
            x\\
            y\\
            z
         \end{pmatrix}
         =
         \begin{pmatrix}
            1\\
            1\\
            1
         \end{pmatrix}.
    \end{equation*}
    
    Podemos mudar as matrizes de parênteses para chaves usando o
    \verb#\begin{bmatrix}# e para barras usando o \verb#\begin{vmatrix}#.
    
    Exemplos:
    
    \begin{equation*}
        \begin{bmatrix}
            1 & 1 & -1\\
            1 & -1 & 1\\
            1 & 1 & 1
        \end{bmatrix}
    \end{equation*}
     
    \begin{equation*}
        \begin{vmatrix}
            1 & 1 & -1\\
            1 & -1 & 1\\
            1 & 1 & 1
        \end{vmatrix}
    \end{equation*}
    
    Para gerar uma matriz genérica $n \times m$, usamos:
    
    \begin{verbatim}
        \begin{equation*}
            \begin{bmatrix}
                a_{11} & a_{12} & \dots & a_{1n} \\
                a_{21} & a_{22} & \dots & a_{2n} \\
                \hdotsfor{4}\\
                a_{m1} & a_{m2} & \dots & a_{mn}
            \end{bmatrix}
        \end{equation*}
    \end{verbatim}
        
    Gerando,
    
    \begin{equation*}
        \begin{bmatrix}
            a_{11} & a_{12} & \dots & a_{1n} \\
            a_{21} & a_{22} & \dots & a_{2n} \\
            \hdotsfor{4}\\
            a_{m1} & a_{m2} & \dots & a_{mn}
        \end{bmatrix}
    \end{equation*}

\subsection{Frações}
    O \LaTeX usa o comando \verb#\frac{X}{Y}# para inserir uma fração no texto.
    O exemplo seria
    
    \begin{equation*}
        \frac{x}{y}
    \end{equation*}
    
    X e Y podem ser qualquer expressão matemática. Por exemplo:
    
    \begin{equation*}
        \frac{x^2 + x + 1}{\frac{y^2}{2}}
    \end{equation*}
    
    Se usarmos o comando \verb#\tfrac{X}{Y}#, geramos uma fração menos que 
    a normal.
    
    \begin{equation*}
        \frac{1}{2}
        >
        \tfrac{1}{2}
    \end{equation*}
    
\section{Expressões de Exemplo}

    \begin{equation*}
        \int_a^b f(x)dx   
    \end{equation*}
    
    \begin{verbatim}
        \begin{equation*}
            \int_a^b f(x)dx   
        \end{equation*}
    \end{verbatim}
    
    \begin{equation*}
        \sum_{i = 1}^3 i   
    \end{equation*}
    
    \begin{verbatim}
        \begin{equation*}
            \sum_{i = 1}^3 i   
        \end{equation*}
    \end{verbatim}
    
    \begin{equation*}
        \binom{n}{1}
    \end{equation*}
    
    \begin{verbatim}
        \begin{equation*}
            \binom{n}{1}
        \end{equation*}
    \end{verbatim}
    
\section{Comandos Customizados}
    O \LaTeX permite a programação de novos comandos. Isso é muito útil,
    quando, por exemplo, estamos escrevendo uma aula de algebra linear e
    a precisamos sempre escrever isso \verb#(x_1,x_2,\dots,x_n)# para
    gerar $(x_1,x_2,\dots,x_n)$, podemos usar o comando
    
    \verb#\newcommand{\nome_do_comando}{comando_a_ser_executado}#
    
    No exemplo anterior, podemos fazer o comando
    
    \verb#\newcommand{\vect}{(x_1,x_2,\dots,x_n)}#
    
    e agora, toda vez que digitarmos \verb#$\vect$# obteremos
    \newcommand{\vect}[3]{(#1_#2,#1_2,\dots,#1_#3)}
    $\vect{x}{1}{n}$
    
    Podemos, ainda, modificar o novo comando para aceitar parâmetros, assim
    não estamos limitados a vetores com coordenadas $X$. O comando ainda é
    o mesmo, mas precisamos modificá-lo um pouco.
    
    \begin{verbatim}
        \newcommand{\vect}[N]{(#1_1,#1_2,\dots,#1_#i)}
        Sendo N o número de argumentos que esse comando vai recerber
        e #i recebe o i-ésimo argumento.
    \end{verbatim}
    
    Logo, o comando    
    
    \begin{verbatim}
        \newcommand{\vect}[3]{(#1_#2,#1_2,\dots,#1_#3)}
    \end{verbatim}
    
    faz um vetor de variável \#1, da posição \#2 até \#3.
    Exemplo,
    
    $\vect{x}{1}{n}$
    
\end{document}
