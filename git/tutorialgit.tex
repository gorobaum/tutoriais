\documentclass[12pt,onecolumn]{article}
\usepackage[brazilian]{babel}
\usepackage[utf8]{inputenc}
\usepackage{amsmath}
\begin{document}

\title{Aula Exercicio de Introducao ao Git}
\author{Thiago de Gouveia Nunes}
\maketitle

\section{Controle de Versão}
    Um controle de versão é um sistema responsável por gravar as mudanças em um
    ou mais arquivos.
    Temos dois tipos de software de controle de versão, os que usão o para
    digma centralizado e o distribuído. O SVN e o CVS são exemplos de sistemas
    centralizados. Eles guardão todos os dados de um projeto em um servidor central,
    online. O git e o mercurial são exemplos de sistemas distribuidos. Eles, além
    de usarem repositórios online, criam um repositório local no seu computador.

\section{O que é o git}
    Enquanto a maioria dos SCV guardam as informações como um arquivo base e as suas
    modificações, o git guarda uma "foto" das suas modificações. Toda vez que você
    modifica ou salva algo no git, ele tira uma "foto" de todo o seu espaço de tra
    balho.
    
    A maioria do trabalho do git é local. O git salva todas as modificações dos arquivos
    localmente, e toda vez que você baixar as atualizações do seu projeto, todo o
    histórico de modificações vem junto.
    
    O git tem integridade. Isso quer dizer que é impossível modificar arquivos sem o
    git saber que você está fazendo algo. O git usa um mecanismo chamado SHA-1 hash.
    O git quase sempre adiciona dados. Na maioria das vezes, o git só adiciona dados
    no repositório. Assim, você pode sempre resgatar arquivos que já foram deletados
    a muito tempo.
    
    O git usa uma mecânica de 3 estagios para controlar os arquivos na sua área de
    trabalho. Estes estagios sao: commited, modified e staged. Commited significa
    que o arquivo esta na sua database local. Modificado quer dizer que o arquivo
    foi modificado. Staged quer dizer que o arquivo foi selecionado para entrar no seu
    próximo commit.

\section{Como iniciar um rep no git}

	O github tem um bom tutorial para instalar e iniciar o git:
	\begin{itemize}
    \item Windows   http://help.github.com/win-set-up-git/ ( Sem o tortoise git. Há um
                    tutorial breve sobre o turtoise git abaixo. )
    \item Linux     http://help.github.com/linux-set-up-git/
    \item MacOS     http://help.github.com/mac-set-up-git/
    \end{itemize}
    
\section{Iniciar o git}
    A primeira coisa a se fazer depois de instalar o git e colocar o seu nome de
usuario e e-mail, assim todo commit que voce fizer estara com essas informacoes.
Para isso, use os comandos:

\verb#$git config --global user.name "Seu Nome"#
\verb#$git config --global user.email seuemail@exemplo.comando#

    Para configurar a sua ferramenta de diff, use o comando:

\verb#$git config --global merge.tool NomeDaFerramenta#

Obs: O git aceita kdiff3, tkdiff, meld, xxdiff, emerge, vimdiff, gvimdiff, ecmerge
e opendiff como ferramentas de merge.

\section{Help no git}
    Para acessar o manual do git ( e de seus comandos ) voce pode usar qualquer
uma das sintaxes abaixo:

\verb#$git help <comando>#
\verb#$git <comando> --help#
\verb#$man git-<comando>#

\section{Como usar o git}
	Agora que o git está instalado e o repositório já está criado, podemos
começar a modificar nosso projeto. Todo arquivo no git tem dois status: Tracked
e Untracked. Arquivos Tracked serão adicionados no próximo commit. Para verifica
o status de cada arquivos usamos o comando git status. O comando git commit adic
iona os arquivos Tracked para o seu repositório. Toda vez que um arquivo é modi
ficado ele é marcado como Untracked. Para mudar o status de um arquivo para 
Tracked usamos o comando git add \verb#<Nome-do-Arquivo>#. O comando git diff mostra a 
diferença entre os arquivos Untracked e os Tracked. Um arquivo pode estar Tracked
e Untracked ao mesmo tempo se dermos git add nele, modificarmos ele, e executar
mos o git status, ele vai mostrar o mesmo arquivo nas duas partes, a que mostra 
os arquivos que serão commitados e os que não serão. Isso quer dizer que a versão
não modificada do seu arquivo vai ser commitada, e não a mais nova. Para corrigir
isso teriamos que dar git add no mesmo arquivo. Para remover um arquivo do git 
usamos o comando git rm <Nome-do-Arquivo>. O comando \verb#git mv file_from file_to#
move um arquivo.
	O git commit só modifica os arquivos no seu repositorio local. Precisamos 
usar o comando git push para mandar as modificações para o repositorio no github.
Se a sua versão for mais nova que a do repositorio, o seu push será feito e as 
modificações serão feitas. Se não, usamos o comando git pull para pegar a versão
mais nova do repositorio e dar um merge com o nosso repositorio. Se houver
algum problema com o merge, por exemplo um arquivo que você modificou foi modi
ficado no repositorio no github, você terá que abrir o arquivo e modifica-lo
manualmente para arrumá-lo. Depois disso, o git push poderá ser executado sem
problema.


\section{Branches}
    Branches sao facilmente criados no git. Eles sao arquivos de 41bits que apon
tam para as mudancas feitas dentro deles. Vale lembrar que um branch criado por
voce e local, e ele so e adicionado no repositorio local se usarmos o push desta
maneira:
    git push (remote) (branch)
exemplo: git push origin teste, isso vai fazer o branch teste ser integrado
no nosso repsitorio remoto origin.

\section{Exercicio}
    Primeiro, precisamos copiar o repositorio remoto para o computador. Usamos o
comando clone para isso. Ele vai copiar inteiramente o repositorio alvo para o 
local onde ele foi invocado. Copiaremos o repositorio que esta em :
    git@github.com:gorobaum/DummyRep.git 
    Agora, vamos mexer nesse repositorio. Primeiramente, crie uma pasta e um ar
quivo de texto, os dois com o seu nome, dentro da pasta DummyRep. Escreva 3 linhas
de texto no arquivo. Use o comando add do git para adicionar a pasta e o arquivo
criados no proximo commit. Use o comando commit para commitar as suas mudancas.
Lembre-se de escrever mensagens curtas mas informativas sobre o conteudo do seu 
commit. Com isso, o seu repositorio local tem a sua pasta, mas o repositorio re
moto ainda nao sabe que essa pasta existe. Para isso, temos que usar o comando
push, que manda todos os commits que voce fez para o repositorio remoto. Use
o push agora. Com isso suas mudancas estao agora dentro do repositorio remoto.
    Agora vamos criar um branch. Crie um branch com o nome teste1. Agora de check
out teste1 para "entrar" no branch teste1 e modifique a segunda linha do seu ar
quivo de texto. Commite as mudancas. Agora volte para o master, e modifique a 
segunda linha do mesmo arquivo, de uma maneira diferente e commite. Tente dar 
merge com o teste1. O merge dara problema, e voce tera que mudar manualmente o 
arquivo. Ao abrir o arquivo, ele estara dividido em duas partes:
    - Uma que vai de <<<<<<<<<< HEAD ate ===========. Essa parte e a que esta no
branch master.
    - Outra que vai de ========== ate >>>>>>>>>> teste1. Essa parte e a que esta
no branch teste1.
Modifique o arquivo ate que ele fique do jeito que voce desejar e de um commit.
Agora de um push. Pronto, voce criou e modificou arquivos de um repositorio 
remoto usando o git.

\section{Resumo dos comandos unix}
\verb#$ git clone -> Copia um repositorio remoto para a atual localização.#
\verb#$ git add <Arquivo> -> Adiciona o arquivo para ser commitado.#
\verb#$ git commit -> Comita as atuais modificações para o seu repositorio local.#
\verb#$ git push -> Mandas os atuais commits para o repositorio remoto.#
\verb#$ git pull -> Puxa os commits do repositorio remoto.#
\verb#$ git rm -> Remove um arquivo.#
\verb#$ git mv -> Move um arquivo.#
\verb#$ git diff -> Mostra a diferença entre os arquivos que serão comitados e suas atuais modificações#
\verb#$ git status -> Mostra o status de cada arquivo.#
\verb#$ git branch <Nome-do-Branch> -> Cria um branch com o nome passado.#
\verb#$ git checkout <Nome-do3-Branch> -> Vai para o branch com o nome passado. O branch tem que existir.#

\section{Git no Windows usando o tortoise git}
    Primeiro precisamos instalar os seguintes arquivos:
    - msysgit       http://code.google.com/p/msysgit/
    - Tortoise Git  http://code.google.com/p/tortoisegit/
    - PuTTY         http://www.chiark.greenend.org.uk/~sgtatham/putty/download.html
Baixe as versoes mais novas, e instale tudo normalmente, sem mudar nenhuma opcao,
a nao ser que seja de sua preferencia.
Agora, precisamos configurar uma chave de ssh para que se computador possa con
versar com o repositorio remoto. Va ate a pasta em que o PuTTY foi instalado e 
abra o executavel puttygen. Clique em Generate, depois digite uma senha no campo
Key passphrase ( essa senha sera pedida toda vez que voce fizer um push ou um 
pull. Nao a esqueca! ). Clique em save private key, e a salve num lugar onde 
voce possa achar facilmente. Abra o site do github e na secao Account settings,
va em SSH Public Keys e adicione a chave de ssh que voce acabou de gerar.
Quando voce for clonar um repositorio, o tortoise vai pedir o seu nome e seu email.
Isso e para que cada commit foito por voce tenha as suas informacoes, assim
o grupo pode saber quem fez o que. Depois, voce tera que fornecer o link da chave 
de ssh para o tortoise, sem isso ele nao pode clonar um repositorio.

\end{document}
