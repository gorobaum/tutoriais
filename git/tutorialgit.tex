\documentclass{article}
\usepackage[brazilian]{babel}
\usepackage[utf8]{inputenc}
\usepackage{amsmath}
\usepackage{hyperref}
\begin{document}

\title{Aula Exercicio de Introducao ao Git}
\author{Thiago de Gouveia Nunes}
\maketitle

\section{Controle de Versão}
    Um controle de versão é um sistema responsável por gravar as mudanças em um
    ou mais arquivos.
    Temos dois tipos de software de controle de versão, os que usão o para
    digma centralizado e o distribuído. O SVN e o CVS são exemplos de sistemas
    centralizados. Eles guardão todos os dados de um projeto em um servidor central,
    online. O git e o mercurial são exemplos de sistemas distribuidos. Eles, além
    de usarem repositórios online, criam um repositório local no seu computador.

\section{O que é o git}
    Enquanto a maioria dos SCV guardam as informações como um arquivo base e as suas
    modificações, o git guarda uma "foto" das suas modificações. Toda vez que você
    modifica ou salva algo no git, ele tira uma "foto" de todo o seu espaço de tra
    balho.
    
    A maioria do trabalho do git é local. O git salva todas as modificações dos arquivos
    localmente, e toda vez que você baixar as atualizações do seu projeto, todo o
    histórico de modificações vem junto.
    O git tem integridade. Isso quer dizer que é impossível modificar arquivos sem o
    
    git saber que você está fazendo algo. O git usa um mecanismo chamado SHA-1 hash.
    O git quase sempre adiciona dados. Na maioria das vezes, o git só adiciona dados
    no repositório. Assim, você pode sempre resgatar arquivos que já foram deletados
    a muito tempo.
    
    O git usa uma mecânica de 3 estagios para controlar os arquivos na sua área de
    trabalho. Estes estagios sao: unmodified, modified e staged. Commited significa
    que o arquivo esta na sua database local. Modificado quer dizer que o arquivo
    foi modificado. Staged quer dizer que o arquivo foi selecionado para entrar no seu
    próximo commit.

\section{Como iniciar um rep no git}

	O github tem um bom tutorial para instalar e iniciar o git:
	\begin{itemize}
    \item \href{http://help.github.com/win-set-up-git/}{Windows} ( Sem o tortoise git. Há um
                    tutorial breve sobre o turtoise git abaixo. ) 
    \item \href{http://help.github.com/linux-set-up-git/}{Linux}
    \item \href{http://help.github.com/mac-set-up-git/}{MacOS} 
    \end{itemize}
    
\section{Iniciar o git}
    A primeira coisa a se fazer depois de instalar o git e colocar o seu nome de
usuario e e-mail, assim todo commit que voce fizer estara com essas informacoes.
Para isso, use os comandos:

\begin{tabbing}
\hspace{1cm}\=   \verb#$git config --global user.name "Seu Nome"# \\
            \>   \verb#$git config --global user.email seuemail@exemplo.comando#
\end{tabbing}
    Para configurar a sua ferramenta de diff, use o comando:

\begin{tabbing}
\hspace{1cm}\=   \verb#$git config --global merge.tool NomeDaFerramenta#
\end{tabbing}

Obs: O git aceita kdiff3, tkdiff, meld, xxdiff, emerge, vimdiff, gvimdiff, ecmerge
e opendiff como ferramentas de merge.

\section{Help no git}
    Para acessar o manual do git ( e de seus comandos ) voce pode usar qualquer
uma das sintaxes abaixo:

\begin{tabbing}
\hspace{1cm}\=  \verb#$git help <comando>#\\
            \>  \verb#$git <comando> --help#\\
            \>  \verb#$man git-<comando>#
\end{tabbing}

\section{Como usar o git}
	Agora que o git está instalado e o repositório já está criado, podemos
começar a modificar nosso projeto. O git usa um mecânismo para controlar as
modificações nos seus arquivos, e usa um sistema um pouco diferente dos outros
controles de versões para mandar essas modificações para o remoto.

\subsection{Iniciando um repositório local git}
    Para iniciar um rep. local em um diretório usamos o comando
\begin{tabbing}
    \hspace{1cm}\=\verb#$ git init#
\end{tabbing}
ou
\begin{tabbing}
    \hspace{1cm}\=\verb#$ git clone <url>#
\end{tabbing}

    O \verb#git init# inicia um repositório sem qualquer ligação com algum remoto,
e o \verb#git clone# pega tudo de um remoto ( a \verb#<url># ) copia para um novo repositório
local e linka esse local com o remoto de origem.

\subsection{Arquivos no Git}
    Todo arquivo, antes de ser adicionado no seu repositório, é reconhecido no git
como \textbf{Untracked}. Ao ser adicionado usando o comando:
\begin{tabbing}
    \hspace{1cm}\=\verb#$ git add <Nome-do-Arquivo>#\\
\end{tabbing}
ele vira \textbf{Tracked} entra num ciclo de 3 fases do git.

    Esse ciclo consiste nas fases \textbf{modified}, \textbf{staged} e 
\textbf{unmodified}. Quando um arquivo é adicionado no git, ele entra como 
\textbf{staged} e será adicionado ao remoto no próximo commit. Ao ser modificado,
um arquivo vai para \textbf{modified}. Ao ser comitado, o arquivo se torna 
\textbf{unmodified}.
    
    Para verificar quais arquivos estão em qual fase usamos o comando    
\begin{tabbing}
    \hspace{1cm}\=\verb#$ git status#
\end{tabbing}
    ele ainda mostra algums comandos para resetar arquivos e outras dicas, falaremos
mais sobre isso depois.
    
Esse arquivo vai aparecer nos dois locais. Na realidade, o arquivo que aparece
    Um arquivo pode ser \textbf{staged} e \textbf{modified} ao mesmo tempo.
Esperimente dar um git add em um arquivo, modificá-lo e dar um git status.
na seção \textbf{modified} é o arquivo modificado, e o arquivo que está na
\textbf{staged} é o arquivo sem modifcação.

    Para verificar as diferenças entre os arquivos que estão modificados e
os suas respectivas cópias no local usamos o comando
\begin{tabbing}
    \hspace{1cm}\=\verb#$ git diff#
\end{tabbing}
    
\subsection{Commits}
    Os commits no git, ao contrário dos controles de versão ao estilo SVN,
é local. Isso quer dizer que tudo o que é commitado não vai diretamente para
o repositório remoto, e sim para o seu repositório local. Os arquivos na área
\textbf{staged} serão adicionados no próximo commit. Um commit contêm a célula
do commit, as informações de quem commitou e uma mensagem sobre o commit. Para
commitar suas mudanças, usamos o comando
\begin{tabbing}
    \hspace{1cm}\=\verb#$ git commit#
\end{tabbing}
    ao executar esse comando, um editor de texto vai abrir, e a mensagem do commit
deve ser escrita.
    
    O git commit aceita duas opções muito usadas
    \begin{itemize}
        \item \verb#$ git commit -a # ele commita tudo que está na área \textbf{modified}
        \item \verb#$ git commit -m "string"# ele gera um commit com mensagem igual a string passada.
            Lembre que a string deve ser passada entre áspas.
    \end{itemize}

\subsubsection{Como os commits são armazenados}
    Os commits são armazenados como uma lista ligada, onde cada célula da lista
é um arquivo de 41 bits ( um HASH para identificar cada commit de maneira única ),
um ponteiro para um estrutura que aponta para todos os arquivos daquele commit e
um ponteiro para o commit pai.

    O commit atual é marcado por um apontador chamado HEAD, que é movido toda vez
que fazemos um commit. Está estrutura faz com que commitar e andar nos commit seja
muito leve, pois só mudamos o valor de um ponteiro, e para carregar os arquivo usamos
um arquivo de 41 bits.

\subsubsection{Boas práticas de commit}
    Quando se trabalha com várias pessoas num projeto, é bom seguir algumas práticas
pra deixar o trabalho de todos mais fácil:
    \begin{itemize}
        \item Teste o seu código antes de dar git push. Sempre deixe o código que 
                está no repositório funcional.
        \item Escreva mensagens de commit claras, pequenas e informativas. Elas devem
                explicar o que você fez nesse commit.
        \item Faça vários commit pequenos. Assim, você não adiciona muita informação
                ao mesmo tempo dentro do repositório, facilitando a volta de commits. % MELHORAR ESSA PARTE @@ %
        \item Quebre commits. Ao ínvez de adicionar todas as suas mudanças direto em um grande commit,
                adicione-ás de uma maneira inteligente, se você modificou dois arquivos muito diferentes,
                use um commit para cada um.
    \end{itemize}

\subsubsection{Mandando commits para o repositório remoto}
    Os seus commits só são mandados para o repositório remoto quando usamos o comando
\begin{tabbing}
    \hspace{1cm}\=\verb#$ git push#
\end{tabbing}
    
    Esse comando vai copiar a sua lista ligada à lista ligada de commits do repositório remoto.
Ele vai procurar o commit em comum mais antigo entre seu rep. e o remoto e tentar dar cópiar seus
commits para o remoto. Se o commit mais antigo não for o último do remoto, ou seja, alguem fez algum
push antes do seu, você deve usar o comando  
\begin{tabbing}
    \hspace{1cm}\=\verb#$ git pull#
\end{tabbing}
para pegar as modificações do remoto e aplicar no local, para só depois você poder dar push.

\subsection{Retornando a estados anteriores}
    A grande vantagem de se usar controle de versão é a possibilidade de trazer
arquivos de volta para versões antigas deles. No git, podemos voltar arquivos
separadamente ou voltar para commits especificos.

\subsubsection{Voltando arquivos}
    Para retornar arquivos para a última versão gravada no git usamos o comando
    \begin{tabbing}
        \hspace{1cm}\=\verb#$ git checkout -- <nome-do-arquivo>#
    \end{tabbing}
mas tome cuidado com esse comando, pois ele sobrescreve o seu arquivo, deletando
as suas modificações.
    
    Para retornar para uma versão especifica de um arquivo, em um branch especifico,
usamos o comando
    \begin{tabbing}
        \hspace{1cm}\=\verb#$ git checkout <nome-da-branch>~n <nome-do-arquivo># \\
                    \>( n é  a diferença do número do commit atual e o alvo ).
    \end{tabbing}
    
    \underline{Um exemplo}:
    \begin{tabbing}
        \hspace{0.5cm}  \= Imagine que você quer voltar o arquivo \emph{pessoa.c} para o\\
                        \> estado dele 3 commits atrás. Você usa o comando\\
                        \> \verb#$ git checkout master~3 pessoa.c#.
    \end{tabbing}
    
\subsubsection{Voltando para commits}
    O comando para se mover na lista ligada de commits é bem parecido com o acima
    \begin{tabbing}
        \hspace{1cm}\=\verb#$ git checkout <nome-da-branch>~n# \\
                    \>( n é  a diferença do número do commit atual e o alvo ).
    \end{tabbing}
    
    A diferença é que como não especificamos um arquivo, voltamos todo o projeto
para o seu estado passado.
\subsection{Comunicação com o remoto}
    Como vimos, usamos os seguintes comandos para mandar/receber dados do remoto:
    \begin{itemize}
        \item \verb#$ git pull -> #Recebe as últimas modificações do remoto.
        \item \verb#$ git push -> #Manda seus commits para o remoto.
        \item \verb#$ git clone <url> -> #Clona um remoto para dentro da pasta local.
        \item \verb#$ git remote show -> #Mostra informações sobre o remoto.
        \item \verb#$ git remote <nome-atual> <novo-nome> ->#Renomeia um remoto.
    \end{itemize}

\section{Branches}

\subsection{O que são branches}
    Usamos uma branch num controle de versão quando queremos separar o fluxo de desenvolvimento
de um projeto. Imagine que temos dois fluxos no projeto, é interesante separarmos eles para que
nossa branch principal ( normalmente a master ) não fique poluída com commits que não acresentão
nada diretamente ao projeto.

\subsection{Branches no git}
    Branches sao facilmente criados no git. Eles não passam de outro apontador para 
commits, tipo o HEAD. O objetivo das branches é criar ambientes de trabalho com o
menor contato possível com a branch master ( a princípal do projeto ) assim
não "estragamos" o projeto com features que não estão prontas ainda. Todo trabalho
que fazemos numa branch fica somente nela até que juntamos elas com outra branch.

\subsection{Criando, deletando e usando Branches}
    O comando para criar branches é
\begin{tabbing}
    \hspace{1cm}\=\verb#$ git checkout -d <nome-da-branch># ou\\
                \>\verb#$ git branch <nome-da-branch>#
\end{tabbing}
    A diferença é que no primeiro caso a branch é criada e o git vai pra ela. No segundo
ela só é criada.

    Para se movimentar pelas branches usamos o comando
\begin{tabbing}
    \hspace{1cm}\=\verb#$ git checkout <nome-da-branch>#
\end{tabbing}
    que vai para a branch passada para ele.

    Para deletar branches, você precisa sair delas e usar o comando
\begin{tabbing}
    \hspace{1cm}\=\verb#$ git branch -d <nome-da-branch>#
\end{tabbing}

\subsubsection{Merge}
    O merge é juntar dois arquivos, ou nesse caso, duas branches. Para juntar as
branches, use o comando checkout para ir até a branch na qual você quer que o resultado
fique e use o comando 
\begin{tabbing}
    \hspace{1cm}\=\verb#$ git merge <nome-da-branch>#
\end{tabbing}
ele vai tentar juntar os arquivos da branch alvo com os da branch atual.


\subsubsection{Branches remotas}

Vale lembrar que um branch criado por você é local, e ele só é adicionado no repositorio 
local se usarmos o push desta maneira:
\begin{tabbing}
    \hspace{1cm}\=\verb#$ git push <remote> <branch>#\\
\end{tabbing} 
Exemplo: 
\begin{tabbing}
    \hspace{1cm}\=\verb#$ git push origin teste#\\
\end{tabbing} 
isso vai fazer o branch teste ser integrado no nosso repositório remoto origin.

\subsection{Algumas boas práticas de branches no git}
    Quando criamos uma branch num projeto onde várias pessoas estão trabalhando, é
bom seguir algumas práticas:
\begin{itemize}
    \item Sempre crie branches com nomes que falem sobre o que o branch é. Se
            você está fez um branch para arrumar algo sobre e-mail, um bom nome
            seria e-mailfix.
    \item Criar uma branch toda vez que você for implementar algo no sistema.
            Isso impede que o projeto seja desenvolvido direto na master.
    \item Só de merge com qualquer outra branch quando a feature do branch 
            estiver pronta e testada. Isso impede que pessoas usem coisas
            da sua branch quando ela ainda não está pronta.
\end{itemize}

\section{Resumo dos comandos no linux}
\begin{tabbing}
\hspace{0.5cm}  \=  \verb#$ git clone -> Copia um repositorio remoto para a atual localização.#\\
                \>  \verb#$ git add <Arquivo> -> Adiciona o arquivo para ser commitado.#\\
                \>  \verb#$ git commit -> Comita as atuais modificações para o seu repositorio local.#\\
                \>  \verb#$ git push -> Mandas os atuais commits para o repositorio remoto.#\\
                \>  \verb#$ git pull -> Puxa os commits do repositorio remoto.#\\
                \>  \verb#$ git rm -> Remove um arquivo.#\\
                \>  \verb#$ git mv -> Move um arquivo.#\\
                \>  \verb#$ git diff -> Diferença entre os arquivos que serão comitados e suas modificações.#\\
                \>  \verb#$ git status -> Mostra o status de cada arquivo.#\\
                \>  \verb#$ git branch <Nome-do-Branch> -> Cria um branch com o nome passado.#\\
                \>  \verb#$ git checkout <Nome-do3-Branch> -> Vai para o branch com o nome passado.#\\
\end{tabbing}

\section{"Malandragens" no git}

    % TODO ( git stash e git commit --amend e git tag e git reset HEAD e coisas assim)

\section{Git no Windows usando o tortoise git}
    Primeiro precisamos instalar os seguintes arquivos:
    \begin{tabbing}
        \hspace{1cm}\=\textbf{Tortoise Git}\quad    \=http://code.google.com/p/tortoisegit/\\
                    \>\textbf{msysgit}              \>http://code.google.com/p/msysgit/\\
                    \>\textbf{PuTTY}                \>http://www.chiark.greenend.org.uk/~sgtatham/putty/download.html
    \end{tabbing}
Baixe as versões mais novas, e instale tudo normalmente, sem mudar nenhuma opcao,
a nao ser que seja de sua preferencia.
Agora, precisamos configurar uma chave de ssh para que se computador possa con
versar com o repositorio remoto. Va ate a pasta em que o PuTTY foi instalado e 
abra o executavel puttygen. Clique em Generate, depois digite uma senha no campo
Key passphrase ( essa senha sera pedida toda vez que voce fizer um push ou um 
pull. Nao a esqueca! ). Clique em save private key, e a salve num lugar onde 
voce possa achar facilmente. Abra o site do github e na secao Account settings,
va em SSH Public Keys e adicione a chave de ssh que voce acabou de gerar.
Quando voce for clonar um repositorio, o tortoise vai pedir o seu nome e seu email.
Isso e para que cada commit foito por voce tenha as suas informacoes, assim
o grupo pode saber quem fez o que. Depois, voce tera que fornecer o link da chave 
de ssh para o tortoise, sem isso ele nao pode clonar um repositorio.

\end{document}
