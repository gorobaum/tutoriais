\documentclass{article}
\usepackage[brazilian]{babel}
\usepackage[utf8]{inputenc}
\usepackage{amsmath}
\begin{document}

\title{Aula Exercicio de Introducao ao Git}
\author{Thiago de Gouveia Nunes}
\maketitle

\section{Controle de Versão}
    Um controle de versão é um sistema responsável por gravar as mudanças em um
    ou mais arquivos.
    Temos dois tipos de software de controle de versão, os que usão o para
    digma centralizado e o distribuído. O SVN e o CVS são exemplos de sistemas
    centralizados. Eles guardão todos os dados de um projeto em um servidor central,
    online. O git e o mercurial são exemplos de sistemas distribuidos. Eles, além
    de usarem repositórios online, criam um repositório local no seu computador.

\section{O que é o git}
    Enquanto a maioria dos SCV guardam as informações como um arquivo base e as suas
    modificações, o git guarda uma "foto" das suas modificações. Toda vez que você
    modifica ou salva algo no git, ele tira uma "foto" de todo o seu espaço de tra
    balho.
    
    A maioria do trabalho do git é local. O git salva todas as modificações dos arquivos
    localmente, e toda vez que você baixar as atualizações do seu projeto, todo o
    histórico de modificações vem junto.
    
    O git tem integridade. Isso quer dizer que é impossível modificar arquivos sem o
    git saber que você está fazendo algo. O git usa um mecanismo chamado SHA-1 hash.
    O git quase sempre adiciona dados. Na maioria das vezes, o git só adiciona dados
    no repositório. Assim, você pode sempre resgatar arquivos que já foram deletados
    a muito tempo.
    
    O git usa uma mecânica de 3 estagios para controlar os arquivos na sua área de
    trabalho. Estes estagios sao: commited, modified e staged. Commited significa
    que o arquivo esta na sua database local. Modificado quer dizer que o arquivo
    foi modificado. Staged quer dizer que o arquivo foi selecionado para entrar no seu
    próximo commit.

\section{Como iniciar um rep no git}

	O github tem um bom tutorial para instalar e iniciar o git:
	\begin{itemize}
    \item Windows   http://help.github.com/win-set-up-git/ ( Sem o tortoise git. Há um
                    tutorial breve sobre o turtoise git abaixo. )
    \item Linux     http://help.github.com/linux-set-up-git/
    \item MacOS     http://help.github.com/mac-set-up-git/
    \end{itemize}
    
\section{Iniciar o git}
    A primeira coisa a se fazer depois de instalar o git e colocar o seu nome de
usuario e e-mail, assim todo commit que voce fizer estara com essas informacoes.
Para isso, use os comandos:

\begin{tabbing}
\hspace{1cm}\=   \verb#$git config --global user.name "Seu Nome"# \\
            \>   \verb#$git config --global user.email seuemail@exemplo.comando#
\end{tabbing}
    Para configurar a sua ferramenta de diff, use o comando:

\begin{tabbing}
\hspace{1cm}\=   \verb#$git config --global merge.tool NomeDaFerramenta#
\end{tabbing}

Obs: O git aceita kdiff3, tkdiff, meld, xxdiff, emerge, vimdiff, gvimdiff, ecmerge
e opendiff como ferramentas de merge.

\section{Help no git}
    Para acessar o manual do git ( e de seus comandos ) voce pode usar qualquer
uma das sintaxes abaixo:

\begin{tabbing}
\hspace{1cm}\=  \verb#$git help <comando>#\\
            \>  \verb#$git <comando> --help#\\
            \>  \verb#$man git-<comando>#
\end{tabbing}

\section{Como usar o git}
	Agora que o git está instalado e o repositório já está criado, podemos
começar a modificar nosso projeto. O git usa um mecânismo para controlar as
modificações nos seus arquivos, e usa um sistema um pouco diferente dos outros
controles de versões para mandar essas modificações para o remoto.

\subsection{Arquivos no Git}
    Todo arquivo, antes de ser adicionado no seu repositório, é reconhecido no git
como \textbf{Untracked}. Ao ser adicionado usando o comando:\\
\begin{tabbing}
\hspace{1cm}\=\verb#$ git add <Nome-do-Arquivo>#\\
\end{tabbing}
ele vira \textbf{Tracked} entra num ciclo de 3 fases do git.

    Esse ciclo consiste nas fases \textbf{modified}, \textbf{staged} e 
\textbf{unmodified}. Quando um arquivo é adicionado no git, ele entra como 
\textbf{staged} e será adicionado ao remoto no próximo commit. Ao ser modificado,
um arquivo vai para \textbf{modified}. Ao ser comitado, o arquivo se torna 
\textbf{unmodified}.
    
    Para verificar quais arquivos estão em qual fase usamos o comando    
\begin{tabbing}
    \hspace{1cm}\=\verb#$ git status#
\end{tabbing}
    ele ainda mostra algums comandos para resetar arquivos e outras dicas, falaremos
mais sobre isso depois.
    
    Um arquivo pode ser \textbf{staged} e \textbf{modified} ao mesmo tempo.
Esperimente dar um git add em um arquivo, modificá-lo e dar um git status.
Esse arquivo vai aparecer nos dois locais. Na realidade, o arquivo que aparece
na seção \textbf{modified} é o arquivo modificado, e o arquivo que está na
\textbf{staged} é o arquivo sem modifcação.

    Para verificar as diferenças entre os arquivos que estão modificados e
os suas respectivas cópias no local usamos o comando
\begin{tabbing}
    \hspace{1cm}\=\verb#$ git diff#
\end{tabbing}

    
\subsection{Commits}
    Os commits no git, ao contrário dos controles de versão ao estilo SVN,
é local. Isso quer dizer que tudo o que é commitado não vai diretamente para
o repositório remoto, e sim para o seu repositório local. Os arquivos na área
\textbf{staged} serão adicionados no próximo commit.

    Os seus commits só são mandados para o repositório remoto quando usamos o comando
\begin{tabbing}
    \hspace{1cm}\=\verb#$ git push#
\end{tabbing}

    Os commits são armazenados como uma lista ligada, onde cada célula da lista
é um arquivo de 41 bits ( um 
    
\section{Branches}
    Branches sao facilmente criados no git. Eles sao arquivos de 41bits que apon
tam para as mudancas feitas dentro deles. Vale lembrar que um branch criado por
voce e local, e ele so e adicionado no repositorio local se usarmos o push desta
maneira:
    git push (remote) (branch)
exemplo: git push origin teste, isso vai fazer o branch teste ser integrado
no nosso repsitorio remoto origin.

\section{Resumo dos comandos no linux}
\begin{tabbing}
\hspace{0.5cm}  \=  \verb#$ git clone -> Copia um repositorio remoto para a atual localização.#\\
                \>  \verb#$ git add <Arquivo> -> Adiciona o arquivo para ser commitado.#\\
                \>  \verb#$ git commit -> Comita as atuais modificações para o seu repositorio local.#\\
                \>  \verb#$ git push -> Mandas os atuais commits para o repositorio remoto.#\\
                \>  \verb#$ git pull -> Puxa os commits do repositorio remoto.#\\
                \>  \verb#$ git rm -> Remove um arquivo.#\\
                \>  \verb#$ git mv -> Move um arquivo.#\\
                \>  \verb#$ git diff -> Diferença entre os arquivos que serão comitados e suas modificações.#\\
                \>  \verb#$ git status -> Mostra o status de cada arquivo.#\\
                \>  \verb#$ git branch <Nome-do-Branch> -> Cria um branch com o nome passado.#\\
                \>  \verb#$ git checkout <Nome-do3-Branch> -> Vai para o branch com o nome passado.#\\
\end{tabbing}

\section{Git no Windows usando o tortoise git}
    Primeiro precisamos instalar os seguintes arquivos:
    \begin{tabbing}
        \hspace{1cm}\=\textbf{Tortoise Git}\quad    \=http://code.google.com/p/tortoisegit/\\
                    \>\textbf{msysgit}              \>http://code.google.com/p/msysgit/\\
                    \>\textbf{PuTTY}                \>http://www.chiark.greenend.org.uk/~sgtatham/putty/download.html
    \end{tabbing}
Baixe as versões mais novas, e instale tudo normalmente, sem mudar nenhuma opcao,
a nao ser que seja de sua preferencia.
Agora, precisamos configurar uma chave de ssh para que se computador possa con
versar com o repositorio remoto. Va ate a pasta em que o PuTTY foi instalado e 
abra o executavel puttygen. Clique em Generate, depois digite uma senha no campo
Key passphrase ( essa senha sera pedida toda vez que voce fizer um push ou um 
pull. Nao a esqueca! ). Clique em save private key, e a salve num lugar onde 
voce possa achar facilmente. Abra o site do github e na secao Account settings,
va em SSH Public Keys e adicione a chave de ssh que voce acabou de gerar.
Quando voce for clonar um repositorio, o tortoise vai pedir o seu nome e seu email.
Isso e para que cada commit foito por voce tenha as suas informacoes, assim
o grupo pode saber quem fez o que. Depois, voce tera que fornecer o link da chave 
de ssh para o tortoise, sem isso ele nao pode clonar um repositorio.

\end{document}
